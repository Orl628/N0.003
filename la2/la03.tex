\subsection{Eigenwerte}
Es ist wünschenswert, lineare Abbildungen mit möglichst einfachen Matrizen
darstellen zu können. In diesem Kapitel werden Endomorphismen eines Vektorraums
betrachtet und eine Basis gesucht, so dass die darstellende Matrix möglichst
einfach (im idealen Fall eine Diagonalmatrix) ist.

\begin{definition} Für einen Endomorphismus $F$  von $V$ heißt ein $\lambda \in
K$ \textbf{Eigenwert} von $F$, falls ein $v \in V \setminus \{ 0 \}$ mit $F(v) =
\lambda \cdot v$ existiert. $v$ heißt in diesem Fall ein \textbf{Eigenvektor}
zum Eigenwert $\lambda$.
\end{definition}

\begin{definition} Ein Endomorphismus heißt \textbf{diagonalisierbar}, wenn eine
Basis aus Eigenvektoren existiert. (Für $\text{dim}V = n < \infty$ ist ein
Endomorphismus $F$ genau dann diagonalisierbar, wenn es eine Basis $\mathcal{B}$
gibt, so dass  $M_{\mathcal{B}}(F)$ eine Diagonalmatrix ist.)
\end{definition}

\begin{lemma} Eigenwerte $w_1$, $\ldots$, $w_m$ zu paarweise verschiedenen
Eigenwerte $\lambda_1$, $\ldots$, $\lambda_m$ sind linear unabhängig.
\end{lemma}
\begin{proof}
Mit Induktion über $m$. Für den Induktionsschritt betrachtet man eine
Linearkombination $\sum_{i=0}^m \mu_i v_i = 0$. Dann gilt
\begin{center}
$0 = \lambda_1 \cdot \sum_{i=0}^m \mu_i v_i - F\left( \sum_{i=0}^m \mu_i v_i
\right) = \sum_{i=0}^m (\lambda_1 \mu_i v_i-\lambda_i \mu_i v_i) =
\sum_{i=1}^m (\lambda_1 - \lambda_i) \mu_i v_i$.
\end{center}
Nun kann man die Induktionsvoraussetzung verwenden.
\end{proof}

\begin{satz} Ein Endomorphismus von $V$ ist diagonalisierbar, wenn er paarweise
verschiedene Eigenwerte $\lambda_1$, $\ldots$, $\lambda_n$ mit $n = \text{dim}
V$ besitzt.
\end{satz}
\begin{proof}
Folgt direkt aus dem obigen Lemma.
\end{proof}

\begin{definition} Für einen Endomorphismus $F$ von $V$ und ein $\lambda \in K$
definieren wir den \textbf{Eigenraum} von $F$ bezüglich $\lambda$ als
$\text{Eig}( F ; \lambda ) := \{ v \in V: F(v) = \lambda \cdot v \}$. (Dies ist
in der Tat ein Unterraum von $V$.)
\end{definition}

\subsection{Das Charakteristische Polynom}

\begin{bemerkung} Folgende sind äquivalent:
\begin{enumerate}[label=\roman*)]
\item $\lambda$ ist Eigenwert von $F$,
\item $\text{Eig}(F; \lambda) = \ker (F - \lambda \cdot \text{id}) \supsetneq
\{ 0 \}$,
\item $\det (F - \lambda \cdot \text {id}) = 0$.
\end{enumerate}
\end{bemerkung}

\begin{definition} Für $A \in \text{M}_n(K)$ heißt $P_A := \det (A - t \cdot
E_n) \in K[t]$ das \textbf{charakteristische Polynom} von $A$, wobei $t$ als
unbestimmte bzw. ein Element von $K[t]$ aufzufassen ist.
\end{definition}

\begin{bemerkung} $\deg P_A = n$. Sei also $P_A = \sum_{i=0}^n \alpha_i t^i $.
Dann gelten:
\begin{enumerate}[label = \roman*)]
\item $\alpha_n = (-1)^n$,
\item $\alpha{n-1} = (-1)^{n-1} \cdot \text{Tr}(A) = (-1)^{n-1} \cdot
\sum_{i=1}^n a_{ii}$,
\item $\alpha_0 = \det A$.
\end{enumerate}
\end{bemerkung}

\begin{lemma} Ähnliche Matrizen haben das gleiche charakteristische Polynom.
\end{lemma}
\begin{proof}
Wegen $S \cdot t \cdot E_n \cdot S^{-1} = t \cdot E_n $ folgt
für $A = S B S^{-1}$
\begin{center}
$\det (B - t \cdot E_n) = \det S \cdot \det (B - t \cdot E_) \cdot (\det S)^{-1}
= \det ( SBS^{-1} - S \cdot t \cdot E_n cdot S^{-1}) =\det ( A - t \cdot E_n)$.
\end{center}
\end{proof}

\begin{definition} Für einen Endomorphismus $F$ definiern wir $P_F :=
P_{M_\mathcal{A}(F)}$ als das \textbf{charakteristische Polynom} von $F$, wobei
$\mathcal{A}$ eine beliebige Basis ist. ($P_F$ ist nach dem obigen Lemma
wohldefiniert.) Die dazugehörige Polynomfunktion nennen wir die \textbf{
charakteristische Funktion} von $F$.
\end{definition}

\begin{bemerkung} $\text{Eig}(A;\lambda) =\text{Lös}(A - \lambda \cdot E_n, 0)$.
\end{bemerkung}

