\documentclass[a4paper,12pt]{scrartcl}
\usepackage[left=2.0cm,right=2.0cm,top=1.6cm,bottom=2.8cm]{geometry}
\usepackage[utf8]{inputenc}
\usepackage[T1]{fontenc}
\usepackage[ngerman]{babel}
\usepackage{amsmath}
\usepackage{amsfonts}
\usepackage{amssymb}
\usepackage{amsthm}
\usepackage{amscd}

\usepackage{fancyhdr}
\usepackage{graphicx}
\usepackage{lastpage}
\usepackage{setspace}
\usepackage{enumitem}


\onehalfspacing

\pagestyle{fancy}
\fancyhf{}
\cfoot{\thepage}

\renewcommand{\headrulewidth}{0.0pt}
\renewcommand{\footrulewidth}{0.0pt}

\setlength{\parindent}{0cm}

\newtheorem*{satz}{Satz}
\newtheorem*{lemma}{Lemma}
\newtheorem*{korollar}{Korollar}
\theoremstyle{definition}
\newtheorem*{definition}{Definition}
\newtheorem{aufgabe}{Aufgabe}[section]
\newtheorem*{beispiel}{Beispiel}
\newtheorem*{bemerkung}{Bemerkung}

\newcommand{\Id}{\operatorname{id}}

\newcommand{\Hom}{\operatorname{Hom}}
\newcommand{\End}{\operatorname{End}}
\newcommand{\Aut}{\operatorname{Aut}}
\newcommand{\Gal}{\operatorname{Gal}}
\newcommand{\GL}{\operatorname{GL}}
\newcommand{\SL}{\operatorname{SL}}
\newcommand{\PGL}{\operatorname{PGL}}


\newcommand{\Kern}{\operatorname{Kern}}
\newcommand{\Bild}{\operatorname{Bild}}

\newcommand{\Skizze}[1]{\begin{center} \includegraphics[width=8cm]{#1} \end{center}}



\newcommand{\TF}{\operatorname{TF}}
\newcommand{\CSA}{\operatorname{CSA}}
\newcommand{\Br}{\operatorname{Br}}

\newcommand{\op}{\operatorname{op}}

\newcommand{\Nrd}{\operatorname{Nrd}}
\newcommand{\N}{\operatorname{N}}


\newcommand{\Char}{\operatorname{char}}

\newcommand{\Inf}{\operatorname{Inf}}
\title{Zusammenfassung Algorithmische Mathematik II}
\begin{document}
\maketitle
% Abschnitt 2.3 Unabhaengigkeit von Ereignissen (Leon)
\subsection{Unabhängigkeit von Ereignissen}

\begin{definition}
Zwei Ereignisse heissen unabhängig, falls 
\begin{center}
$P[A \cap B]=P[A] \cdot P[B]$
\end{center}
gilt. \\
Eine beliebige (nicht notwendig endlich oder abzählbar!) Kollektion von Ereignissen $A_i$ ($i \in I$) heisst unabhängig, falls 
\begin{center}
$P[A_{i_1} \cap ... \cap A_{i_n}]= \prod_{k=1}^n P[A_{i_k}]$ für alle $n \in \mathbb{N}$ und alle paarweise verschiedenen $i_1, ... , i_n \in I$ gilt.
\end{center}
\end{definition}

\begin{satz}
Sind die Ereignisse $A_1, ... , A_n \in A$ unabhängig und $B_j=A_j$ oder $B_j={A_j}^C$ für alle $j \in \{1, ... , n\}$, so sind auch die Ereignisse $B_1, ... , B_n$ unabhängig.
\end{satz}

Seien $A_1, A_2, ...$ unabhängige Ereignisse mit jeweils Wahrscheinlichkeit $p$.  Wir definieren die Wartezeit auf das erste Eintreten eines Ereignisses durch
\begin{center}
$T(\omega)=\min \{n \in \mathbb{N} : \omega \in A_n \}$
\end{center}
. \\
Es gilt $P[T=n]=p \cdot (1-p)^{n-1}$.

\begin{definition}
Die Wahrscheinlichkeitsverteilung auf $\mathbb{N}$ mit Massenfunktion 
\begin{center}
$p(n)=p \cdot (1-p)^{n-1}$
\end{center}
heisst geometrische Verteilung zum Parameter $p$.
\end{definition}

Die Wahrscheinlichkeit, dass unter $n$ Ereignissen $k$ eintreten ist gleich der Binomialverteilung.

Sei $S_n$ gleich der Anzahl der eingetretenen Ereignisse innerhalb der ersten $n$ Ereignisse.
\begin{satz}
(Bernstein-Ungleichung) \\
\begin{center}
$\forall \epsilon > 0 \forall n \in \mathbb{N} P[\frac{S_n}{n} \geq p + \epsilon] \leq e^{-2 \epsilon^2 n}$
\end{center}
(analog für $\geq p - \epsilon$)
\end{satz}

\end{document}
