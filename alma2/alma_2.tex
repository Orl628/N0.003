\documentclass[a4paper,12pt]{scrartcl}
\usepackage[left=2.0cm,right=2.0cm,top=1.6cm,bottom=2.8cm]{geometry}
\usepackage[utf8]{inputenc}
\usepackage[T1]{fontenc}
\usepackage[ngerman]{babel}
\usepackage{amsmath}
\usepackage{amsfonts}
\usepackage{amssymb}
\usepackage{amsthm}
\usepackage{amscd}

\usepackage{fancyhdr}
\usepackage{graphicx}
\usepackage{lastpage}
\usepackage{setspace}
\usepackage{enumitem}


\onehalfspacing

\pagestyle{fancy}
\fancyhf{}
\cfoot{\thepage}

\renewcommand{\headrulewidth}{0.0pt}
\renewcommand{\footrulewidth}{0.0pt}

\setlength{\parindent}{0cm}

\newtheorem*{satz}{Satz}
\newtheorem*{lemma}{Lemma}
\newtheorem*{korollar}{Korollar}
\theoremstyle{definition}
\newtheorem*{definition}{Definition}
\newtheorem{aufgabe}{Aufgabe}[section]
\newtheorem*{beispiel}{Beispiel}
\newtheorem*{bemerkung}{Bemerkung}

\newcommand{\Id}{\operatorname{id}}

\newcommand{\Hom}{\operatorname{Hom}}
\newcommand{\End}{\operatorname{End}}
\newcommand{\Aut}{\operatorname{Aut}}
\newcommand{\Gal}{\operatorname{Gal}}
\newcommand{\GL}{\operatorname{GL}}
\newcommand{\SL}{\operatorname{SL}}
\newcommand{\PGL}{\operatorname{PGL}}


\newcommand{\Kern}{\operatorname{Kern}}
\newcommand{\Bild}{\operatorname{Bild}}

\newcommand{\Skizze}[1]{\begin{center} \includegraphics[width=8cm]{#1} \end{center}}



\newcommand{\TF}{\operatorname{TF}}
\newcommand{\CSA}{\operatorname{CSA}}
\newcommand{\Br}{\operatorname{Br}}

\newcommand{\op}{\operatorname{op}}

\newcommand{\Nrd}{\operatorname{Nrd}}
\newcommand{\N}{\operatorname{N}}


\newcommand{\Char}{\operatorname{char}}

\newcommand{\Inf}{\operatorname{Inf}}
\title{Zusammenfassung Algorithmische Mathematik II}
\begin{document}
\maketitle
\subsection{Zufallsvariablen und ihre Verteilung}
\begin{definition}
\begin{itemize}
\item Eine \textbf{diskrete Zufallsvariable} ist eine Abbildung
\[X:\Sigma \rightarrow S,\]
wobei $S$ abz\"ahlbar sei.
\item Die \textbf{Verteilung} von $X$ ist die Wahrscheinlichkeitsverteilung $\mu_X$ auf $S$ mit Gewichten
\[p_X(a):=P[X^{-1}(a)]\]
\end{itemize}
\end{definition}
\begin{bemerkung}
Wir scheiben $\{X=a\}$ f\"ur $X^{-1}(a)$ und $P[X=a]$ statt $P[\{X=a\}]$.\\
Ist $A\subseteq S$, so kann $\mu x(A)$ als die Wahrscheinlichkeit interpretiert werden, mit der ein Element aus $A$ ausgespuckt wird.
\end{bemerkung}
\subsection{Binomialverteilung}
Motivation: Man zieht eine Kugel aus einer Urne mit $m$ Kugeln und legt sie wieder zur\"uck. Das macht man $n$ mal. Das mathematische Modell sieht folgenderma\ss en aus:\\
Die Kugeln werden mit $1,2,...,m$ durchnummeriert. Die Menge dieser Kugeln sei $S:=\{1,2,...,m\}$\\
Der Ereignisraum ist dann $\Omega=S^n$, ein Elementarereignis ist dann $(x_1,x_2,...,x_n)=\omega\in\Omega$, wobei $x_i\in S\forall i\in\{1,2,...,n\}$. (das sind die einzelnen Kugeln)\\
Es wird angenommen, dass die $\omega$ gleichverteilt sind, dh. f\"ur alle $\omega,\omega'\in\Omega$ gilt $p(\omega)=p(\omega')=\frac 1{\lvert S\rvert^n}$.\\
Die Funktion $X_i:\Omega\rightarrow S:\omega=(x_1,x_2,...,x_n)\rightarrow x_i$ gibt das $i$-te Ereignis zur\"uck, also die Kugel, die als $i$-tes gezogen wurde.\\
Sei $E\subseteq S$ ein Teil der $m$ Kugeln mit einer besonderen Eigenschaft (schwarze Kugeln, etc.). Die Wahrscheinlichkeit, dass beim $i$-ten Zug eine solche Kugel gezogen wird, ist gerade
\[P[x_i\in E]=\mu_{X_i}(E)=\frac{\lvert E\rvert}{\lvert S\rvert}=:p,\]
was als Erfolgswahrscheinlichkeit bezeichnet wird.\\
Die Wahrscheinlichkeit, dass dieser Erfolg $k$ mal eintritt, wobei $k\in\{1,...,n\}$, ist
\[P[N=k]=\binom nkp^k(1-p)^{n-k}=:p_{n,p}(k)\]
Ist dies die Massenfunktion einer Wahrscheinlichkeitsverteilng auf $\{0,...,n\}$, so hei\ss t diese Verteilung \textbf{Binomialverteilung} mit Parameter $n$ und $p$. Sie gibt aus, mit welcher Wahrscheinlichkeit bei $n$-maligem Ziehen aus einer Urne genau $k$ mal ein Erfolg gezogen wird.
F\"ur kleine Erfolgswahrscheinlichkeiten $\frac\lambda n$ und gro\ss e $n$ n\"ahert sich die Binomialverteilung an die \textbf{Poissonverteilung} mit Parameter $\lambda$ an:
\[p(k):=\frac{\lambda^k}{k!}e^{-\lambda}=\lim_{n\rightarrow\infty}p_{n,\frac\lambda n}(k)\]
\subsection{Hypergeometrische Verteilung}
Motivation: Man zieht eine Kugel aus einer Urne mit $m$ Kugeln ($r$ rote, $m-r$ schwarze) und legt sie nicht wieder zur\"uck. Das macht man $n$ mal. Das mathematische Modell ist im Wesentlichen analog zur Binomialverteilung. F\"ur die Ereignisse gilt diesmal zus\"atzlich $x_i\neq x_j\forall i,j\in\{1,...,m\}$. $N(\omega):=\text{Anzahl der roten Kugeln in }\omega$. Die Wahrscheinlichkeit, dass $N(\omega)=k$ ist ($k$ rote Kugeln in $\omega$), ist
\[P[N=k]=\frac{\binom rk\binom{m-r}{n-k}}{\binom mn}\]
f\"ur $k\in\{0,...,n\}$. Diese Verteilung hei\ss t \textbf{hypergeometrische Verteilung} mit Parametern $m,r,n$.\\
F\"ur $n\rightarrow\infty$ n\"ahert sie sich an die Binoialverteilung an:
\[P[N=k]\rightarrow\binom nk p^k(1-p)^k\]
%%%
%%%
%%%
%%%

%Qi Cheng Hua
% Abschnitt 2.3 Unabhaengigkeit von Ereignissen (Leon)
\subsection{Unabhängigkeit von Ereignissen}

\begin{definition}
Zwei Ereignisse heissen unabhängig, falls 
\begin{center}
$P[A \cap B]=P[A] \cdot P[B]$
\end{center}
gilt. \\
Eine beliebige (nicht notwendig endlich oder abzählbar!) Kollektion von Ereignissen $A_i$ ($i \in I$) heisst unabhängig, falls 
\begin{center}
$P[A_{i_1} \cap ... \cap A_{i_n}]= \prod_{k=1}^n P[A_{i_k}]$ für alle $n \in \mathbb{N}$ und alle paarweise verschiedenen $i_1, ... , i_n \in I$ gilt.
\end{center}
\end{definition}

\begin{satz}
Sind die Ereignisse $A_1, ... , A_n \in A$ unabhängig und $B_j=A_j$ oder $B_j={A_j}^C$ für alle $j \in \{1, ... , n\}$, so sind auch die Ereignisse $B_1, ... , B_n$ unabhängig.
\end{satz}

Seien $A_1, A_2, ...$ unabhängige Ereignisse mit jeweils Wahrscheinlichkeit $p$.  Wir definieren die Wartezeit auf das erste Eintreten eines Ereignisses durch
\begin{center}
$T(\omega)=\min \{n \in \mathbb{N} : \omega \in A_n \}$
\end{center}
. \\
Es gilt $P[T=n]=p \cdot (1-p)^{n-1}$.

\begin{definition}
Die Wahrscheinlichkeitsverteilung auf $\mathbb{N}$ mit Massenfunktion 
\begin{center}
$p(n)=p \cdot (1-p)^{n-1}$
\end{center}
heisst geometrische Verteilung zum Parameter $p$.
\end{definition}

Die Wahrscheinlichkeit, dass unter $n$ Ereignissen $k$ eintreten ist gleich der Binomialverteilung.

Sei $S_n$ gleich der Anzahl der eingetretenen Ereignisse innerhalb der ersten $n$ Ereignisse.
\begin{satz}
(Bernstein-Ungleichung) \\
\begin{center}
$\forall \epsilon > 0 \forall n \in \mathbb{N} P[\frac{S_n}{n} \geq p + \epsilon] \leq e^{-2 \epsilon^2 n}$
\end{center}
(analog für $\geq p - \epsilon$)
\end{satz}

%%%
%%%
%%%
%%%
\subsection{Unabh\"angige Zufallsvariablen und Random Walk}
\begin{definition}
Seien $X_i:\Omega\rightarrow S_i,i\in\{1,...,n\}$ diskrete Zufallsvariablen auf dem Wahrscehinlichkeitsraum $(\Omega,A,P)$. Dann ist $(X_1,...,X_n):\Omega\rightarrow S_1\times...\times S_n$ eine Zufallsvariable.\\
Die Verteilung des Zufallsvektors $(X_1,...,X_n)$ hei\ss t \textbf{gemeinsame Verteilung} der Zufallsvariablen $X_1,...,X_n$. Ihre Mssenunktion ist
\[p_{X_1,...,X_n}(a_1,...,a_n)=P[X_1=a_1,...,X_n=a_n].\]
Die diskreten Zufallsvariablen $(X_1,...,X_n)$ hei\ss en \textbf{unabh\"angig}, falls gilt
\[P[X_1=a_1,...,X_n=a_n]=\prod_{i=1}^nP[x_i=a_i]\forall a_i\in S_i, i\in\{1,...,n\}\]
Unendlich viele diskrete Zufallsvariablen $X_i:\Omega\rightarrow S_i, i\in I$ hei\ss en \textbf{unabh\"angig}, falls die Ereignisse $\{X_i=a_i\},i\in I$ f\"ur alle $a_i\in S$ unabh\"angig sind.
\end{definition}
\begin{satz}
Die folgenden Aussagen sind \"aquivalent:
\begin{itemize}
\item $X_1,...,X_n$ sind unabh\"angig.
\item $p_{X_1,...,X_n}(a_1,...,a_n)=\prod_{i=1}^np_{X_i}(a_i)$
\item $\mu _{X_1,...,X_n}=\mu_{X_1}\times...\times\mu_{X_n}$
\item Die Ereignisse $\{X_1\in A_1\},...,\{X_n\in A_n\}$ sind unabh\"angig f\"ur alle $A_i\subseteq S_i, i\in\{1,...,n\}$.
\item Die Ereignisse $\{X_1=a_1\},...,\{X_1=a_1\}$ sind unabh\"angig f\"ur alle $a_i\in S_i,i\in\{1,...,n\}$
\end{itemize}
Dabei wird immer wieder dieselbe Aussage getroffen, f\"ur einzelne Werte der Zufallsvariablen, oder f\"ur Teilmengen von Werten der Zufallsvariablen.
\end{satz}
\subsubsection{Der Random Walk auf $\mathbb Z$}
Wir laufen auf der Zahlengeraden mit ganzzahligen Eintr\"agen, beginnend beim Startwert $a$, mit Wahrscheinlichkeit $p$ um 1 vorw\"arts und mit Wahrscheinlichkeit $1-p$ um 1 r\"uckw\"arts. Die mathematische Modellierung ist wie folgt:
Der Ereignisraum $\Omega$ ist die Menge aller Random Walks, also alle Folgen $(S_i)_{i\in\mathbb N}$, mit $S_0=a\in\mathbb Z$ und $\lvert S_j-S_{j+1}\rvert=1\forall j\in\mathbb N$.\\
Der $i-1$-te Schritt wird durch die Zufallsvariable $X_i:\Omega\rightarrow \{-1,+1\}$ angegeben. Es gilt $P[X_i=+1]=p, P[X_i=-1]=1-p\forall i\in\mathbb N\backslash\{0\}, p\in(0,1)$. Dann gilt $S_0=a,S_{n+1}=S_n+X_{n+1}$. Induktiv folgt $S_n=a+\sum_{i=1}^nX_i$.\\
Klar ist, dass man in einer geraden Anzahl von Schritten stets ein Element aus $a+2\mathbb Z$ erreicht und in einer ungeraden Anzahl von Schritten stets ein Element aus $a+1+2\mathbb Z$ erreicht. Es gilt
\begin{lemma}
Sei $k\in\mathbb Z$. Dann gilt
\[P[S_n=a+k]=\left\{\begin{array}{ll} 0&falls\ n+k\ ungerade\ oder\ \lvert k\rvert>n,\\ \binom n{\frac{n+k}2}p^{\frac{n+k}2}(1-p)^{\frac{n-k}2}&sonst.\end{array}\right.\]
\end{lemma}
\subsubsection{Symmetrischer Random Walk}
Wir betrachten nun den Fall $p=\frac 12$.\\
Sei $\lambda\in\mathbb Z$ fest. Wir definieren die Zufallsvariable
\[T_\lambda:\Omega\rightarrow mathbb N\cup\{\infty\}:\omega\mapsto T_\lambda(\omega):=\inf\{n\in\mathbb N\backslash\{0\}\mid S_n(\omega)=\lambda\}\]
$T_\lambda(\omega)$ gibt den Zeitpunkt aus, an dem $\lambda$ zum ersten Mal in $\omega$ erreicht wird. Wir wollen $P[T_\lambda\leq n]=P\left[\\bigcup_{i=1}^{n}\{S_i=\lambda\}\right]$ berechnen, die Wahrscheinlichkeit, dass $\lambda$ innerhalb der ersten $n$ Schritte erreicht wird.\\
Nach $n$ Schritten abbrechende Random Walks k\"onnen als Folgen mit $n$ Folgenglieder interpretiert werden, wobei wieder $S_0=a\in\mathbb Z$ und $\lvert S_j-S_{j+1}\rvert=1\forall j\in\{0,...,n-1\}$ gilt. Bei gegebenem Startwert $a$ gibt es genau $2^n$ verschiedene solche Random Walks. Jede solche Folge tritt dabei mit gleicher Wahrscheinlichkeit auf.
\begin{satz}
Reflektionsprinzip: Seien $\lambda, b\in\mathbb Z$. Es gelte entweder ($a<\lambda$ und $b\leq\lambda$) oder ($a>\lambda$ und $b\geq\lambda$) (dh. $a$ und $b$ liegen beide rechts oder beide links von $\lambda$). Dann gilt:
\[P[T_\lambda\leq n,S_n=b]=P[S_n=b^\ast],\]
wobei $b^\ast:=2\lambda-b$ die Spiegelung von $b$ an $\lambda$ ist. (Dann muss ja $2\lambda=b+b^\ast$ gelten)
\end{satz}
Der Satz besagt also, dass wenn man bereits $\lambda$ erreicht hat, dann ist die Wahrscheinlichkeit, dass man nach einem beliebigen Schritt insgesamt $k$ Schritte vorw\"arts gegangen ist, gleich der Wahrscheinlichkeit, dass man nach einem beliebigen Schritt insgesamt $k$ Schritte r\"uckw\"arts gegangen ist.
\begin{satz}
(Trefferzeitenverteilung) Wir erinnern, dass $a$ der Startwert des Random Walks ist. Es gilt
\begin{itemize}
\item \[P[T_\lambda\leq n]=\left\{\begin{array}{ll} P[S_n\geq\lambda]+P[S_n>\lambda],&falls\ \lambda>a\\ P[S_n\leq\lambda]+P[S_n<\lambda],&falls\ \lambda<a\end{array}\right.\]
\item \[P[T_\lambda=n]=\left\{\begin{array}{ll}
\frac 12P[S_{n-1}=\lambda-1]-\frac 12P[S_{n-1}=\lambda+1],&falls\ \lambda>a\\
\frac 12P[S_{n-1}=\lambda+1]-\frac 12P[S_{n-1}=\lambda-1],&falls\ \lambda<a\end{array}\right.=\left\{\begin{array}{ll}
\frac{\lambda-a}n\binom n{\frac{n+\lambda-a}2}2^{-n},&falls\ \lambda>a\\
\frac{a-\lambda}n\binom n{\frac{n+\lambda-a}2}2^{-n},&falls\ \lambda<a\end{array}\right.\]
\end{itemize}
\end{satz}
\begin{korollar}
(Verteilung des Maximums) Sei $M_n:=\max\{S_0,...,S_n\}$. F\"ur $\lambda>a$ gilt
\[P[M_n\geq\lambda ]=P[T_\lambda\leq n]=P[S_n\geq\lambda]+P[S_n>\lambda]\]
\end{korollar}
%Qi Cheng Hua
\end{document}
